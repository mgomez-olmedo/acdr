\documentclass{article}
%
\usepackage{fullpage}
\usepackage{acdr}
\usepackage[authoryear,round]{natbib}
\bibliographystyle{plainnat}

\newcommand{\noFootnote}[1]{{\small (\textit{#1})}}
\newcommand{\myOp}[1]{{$\left\langle\ensuremath{#1}\right\rangle$}}

\title{Bayesian approach to abrupt concept drift detection}
\author{Cano, A., Gomez-Olmedo, M., Moral, S.
  \\\email{\{acu, mgomez, smc\}@decsai.ugr.es}}
\date{\today}
%

\usepackage{Sweave}
\begin{document}
\Sconcordance{concordance:acdr.tex:acdr.Rnw:%
1 16 1 1 0 13 1 1 5 108 1 1 2 4 0 1 2 29 1 1 2 4 0 1 2 46 1 1 2 4 %
0 1 2 41 1 1 2 4 0 1 2 15 1 1 2 4 0 1 2 15 1 1 2 4 0 1 2 22 1 1 2 %
4 0 1 2 42 1 1 2 4 0 1 2 16 1 1 2 4 0 1 2 88 1}

\maketitle


\setkeys{Gin}{width=\textwidth}

\begin{abstract}
This vignette describes the software developed for the experimental work
presented in the paper \textit{A Bayesian Approach to Abrupt Concept Drift}
in order to allow the reproduction of the experiments.
\end{abstract}

%% Note: These are explained in '?RweaveLatex' :

\section{Introduction}

This vignette is included to descrive the \textit{R} package developed for
performing the experimental work of the paper \textit{A Bayesian Approach to
Abrupt Concept Drift}. The package is named \textit{acdr (Abrupt Concept Drift
in R)} and allows the user to perform all the experiments included in the
paper as well as define new estimation methods and perform new comparisons.

\section{Package structure}

The package is organised in a set of folders commented in the following subsections.

\subsection{R folder}
\textbf{R} folder contains \textbf{R} files with the code of the package. The files
included in it are:

\begin{itemize}
\item \textbf{utils.R}: utility functions for the methods of estimation.

\item \textbf{execution.R}: functions related to the execution of experiments.

\item \textbf{estimateXX.R}: code for the the different algorithms of estimation
considered for the paper where $XX$ is the number identifying the particular
method and in some cases the acronym used to identify the method in the paper.
These files contains two different functions: \textbf{estimateXX} and
\textbf{sexpXX}. The first one implements the real method of estimation having
as arguments the stream to analyze and the concrete values for its parameters.
The second one allows to execute a batch of tests varying some of the parameters
in a certain interval and using several cores in parallel in some cases.

\item \textbf{email.R}: sentences required for the analysis of email data.
\end{itemize}

\subsection{data folder}

This folder contains the \textbf{csv} file with emails data.

\subsection{extdata folder}

It contains the description of the experiments to perform with each estimation
method. The files in this folder are named using the format: $X-experiment$
and $X-experiment-test$. $X$ number denotes the method of estimation. The set
of files with names $X-experiment$ contain the complete description of the
experiments while the second set ($X-experiment-test$) contain simplified
versions just for testing the correct execution of the algorithms.

\medskip

The structure of these files is:

\begin{itemize}
\item the first row contains a number with the length of the data sequence
to generate and how many changes in the sequence will be considered. Therefore
a random value of $p$ will be used and a sequence with the given length will
be generated using this value. This procedure will be repeated as many times
as indicated by the second number. As an example a content $1000, 4$ for this
first line will produce a complete sequence of $4000$ data where each subsequence
of $1000$ corresponds to a given value of $p$.

\item the second line contains the number of repetitions to perform for each
estimation, in order to obtain accurate results being independent of the particular
sequence generated

\item from the third line on follows a description of particular tests. As an
example let us consider the file $1-experiment$ for $estimate1$ method. This
method required two parameters: a window size and a value for $\alpha$ parameter.
Therefore, the line $1, 5, 40, 0.01$ specifies: estimation method ($1$),
a first value of window size to consider ($5$), a last value of window size to
analyze ($40$) and $\alpha$ value ($0.01$).
\end{itemize}

The particular configuration for each estimation method will be described below.

\subsection{results folder}

This folder will contain the results obtained from each estimation. This allows
to store the results and to perform the experiments through different sessions
as well as analyzing the results in a posterior step. Result files follow the
format:

\begin{itemize}
\item identifier of estimation mehod (a number)
\item identifier of the iteration (a number)
\item a list of parameter names and values (pairs of strings and values)
\item \textbf{res} extension
\end{itemize}

For example, the file named $1-45-n-11-alpha-0.05.res$ will contain the result
of the 45th iteration with \textbf{estimate1} with a value of $11$ for $n$
parameter and $0.05$ value for $\alpha$ parameter.

\section{Estimation methods and parameters}

\subsection{ADWIN}

This the algorithm considered as the state of the art for concept drift detection. This
method uses a single parameter named $\delta$. The experiment considers the best value
for this parameter, $0.2$, as indicated in the corresponding paper. This is showed in
the file \textbf{11-experiment}:

\begin{verbatim}
1000,4
100
11, 0.2
\end{verbatim}

The command for making the experiments is:

\begin{Schunk}
\begin{Sinput}
> experiment("../extdata/11-experiment")
\end{Sinput}
\end{Schunk}

\subsection{BAF}

This algorithm requires two arguments: $n$ (window size) and $\alpha$ (value to used
for statistical tests). The experiments performed for this method consider values of
$n$ from $90$ to $110$ and $\alpha$ values going from $0.0001$ to $0.01$. The file
defining the configuration for the execution of this algorithm is included below
(it is stored under \textbf{extdata} folder and named \textbf{34-experiment}).

\begin{verbatim}
1000,4
100
34, 90, 110, 0.0001
34, 90, 110, 0.0002
34, 90, 110, 0.0003
34, 90, 110, 0.0004
34, 90, 110, 0.0005
34, 90, 110, 0.0006
34, 90, 110, 0.0007
34, 90, 110, 0.0008
34, 90, 110, 0.0009
34, 90, 110, 0.001
34, 90, 110, 0.003
34, 90, 110, 0.005
34, 90, 110, 0.008
34, 90, 110, 0.01
\end{verbatim}

The command to execute for generating the result files for this algorithm is:

\begin{Schunk}
\begin{Sinput}
> experiment("../extdata/34-experiment")
\end{Sinput}
\end{Schunk}

The experiments included in the paper consider the following parameterization:

\begin{itemize}
\item \textbf{BFA01} for $n=100$ and $\alpha=0.01$
\item \textbf{BFA001} with $n=100$ and $\alpha=0.001$
\item \textbf{BFA0001} with $n=100$ and $\alpha=0.0001$
\end{itemize}

Note: the path for the experiment file should change depending on the base folder
used for running it. This comment can be applied to the rest of algorithms described
in this vignette.

\subsection{BFV1}

The algorithm requires the following parameters:

\begin{itemize}
\item $n$ for window size.
\item $\alpha_1$ and $\alpha_2$ defines an interval for performing statistical tests.
\item $k$ as the number of samples to discard as a result of the forgetting process.
\end{itemize}

The file called \textbf{35-experiment} contains the parameters
used for testing this algorithm.

\begin{verbatim}
1000,4
100
35, 90, 110, 0.0001, 0.01, 10
35, 90, 110, 0.0001, 0.01, 10
35, 90, 110, 0.0001, 0.01, 10
35, 90, 110, 0.0001, 0.01, 10
35, 90, 110, 0.0001, 0.01, 10
35, 90, 110, 0.0001, 0.01, 10
35, 90, 110, 0.0001, 0.01, 10
35, 90, 110, 0.0001, 0.01, 10
35, 90, 110, 0.0001, 0.01, 10
35, 90, 110, 0.0001, 0.01, 10
35, 90, 110, 0.0001, 0.01, 10
35, 90, 110, 0.0001, 0.01, 10
35, 90, 110, 0.0001, 0.01, 10
35, 90, 110, 0.0001, 0.01, 10
\end{verbatim}

The command to execute for getting the corresponding result files is:

\begin{Schunk}
\begin{Sinput}
> experiment("../extdata/35-experiment")
\end{Sinput}
\end{Schunk}

The experiments of the paper consider the following parameterization:

\begin{itemize}
\item values for $\alpha$ between $0.0001$ and $0.01$, window size
of $n=100$ and a number of samples to forget $k=10$ and an uniform prior distribution
for $\alpha$.
\end{itemize}

\subsection{BFV2}

This algorithm uses the same parameters as the previous one:

\begin{itemize}
\item $n$ for window size.
\item $\alpha_1$ and $\alpha_2$ defines an interval for performing statistical tests.
\item $k$ as the number of samples to discard as a result of the forgetting process.
\end{itemize}

The experiments for this algorithm are specified in the file \textbf{36-experiment}:

\begin{verbatim}
1000,4
100
36, 90, 110, 0.0001, 0.01, 10
36, 90, 110, 0.0002, 0.01, 10
36, 90, 110, 0.0003, 0.01, 10
36, 90, 110, 0.0004, 0.01, 10
36, 90, 110, 0.0005, 0.01, 10
36, 90, 110, 0.0006, 0.01, 10
36, 90, 110, 0.0007, 0.01, 10
36, 90, 110, 0.0008, 0.01, 10
36, 90, 110, 0.0009, 0.01, 10
36, 90, 110, 0.001, 0.01, 10
36, 90, 110, 0.003, 0.01, 10
36, 90, 110, 0.005, 0.01, 10
36, 90, 110, 0.008, 0.01, 10
36, 90, 110, 0.01, 0.01, 10
\end{verbatim}

The comnand for executing the set of experiments is:

\begin{Schunk}
\begin{Sinput}
> experiment("../extdata/36-experiment")
\end{Sinput}
\end{Schunk}

\subsection{FW}

This algorithm presents a single argument defining the size of the active
window. The selection of the best value for this parameter was achieved
observing the results of the executions included in the file \textbf{estimate4-FW.R}:

\begin{verbatim}
1000,4
100
4, 5, 80
\end{verbatim}

The best value is $68$ as showed in the paper. The execution of the experiments
is launched with the following command:

\begin{Schunk}
\begin{Sinput}
> experiment("../extdata/4-experiment")
\end{Sinput}
\end{Schunk}

\subsection{FF}

This method considers a parameter named $\rho$ defining the forgetting factor
of the algorithm. The file used for testing the behavior of the algorithm is
\textbf{3-experiment}:

\begin{verbatim}
1000,4
100
3, 0.8, 0.81, 0.82, 0.83, 0.84, 0.85, 0.86, 0.87, 0.88, 0.89, 0.9, 0.91,
0.92, 0.93, 0.94, 0.95, 0.96, 0.97, 0.98, 0.99, 0.999, 0.9999, 1
\end{verbatim}

These experiments can be carried out with this command:

\begin{Schunk}
\begin{Sinput}
> experiment("../extdata/3-experiment")
\end{Sinput}
\end{Schunk}

The best value estimated by the experinments is $0.97$.

\subsection{SWB}

This method uses two parameters:

\begin{itemize}
\item $n$: size of active window.
\item $\alpha$: value used for the statistical tests.
\end{itemize}

The file with the definition of the experiments performed for this method
is \textbf{7-experiment} (stored if \textbf{extdata} folder):

\begin{verbatim}
1000,4
100
7, 5, 40, 0.01
\end{verbatim}

The experiments for the algorithm are executed with the command:

\begin{Schunk}
\begin{Sinput}
> experiment("../extdata/7-experiment")
\end{Sinput}
\end{Schunk}

Our experiments show that the best value for the significance level of the tests
(second parameter) is $0.04$.

\subsection{SWF}

As the previous algorithm, this method uses two parameters:

\begin{itemize}
\item $n$: size of active window.
\item $\alpha$: value used for the statistical tests.
\end{itemize}

The file named \textbf{8-experiment} defines the experiments performed for
testing the optimal values for the parameters: $28$ and $0.006$.

\begin{verbatim}
1000,4
100
8, 5, 40, 0.001
8, 5, 40, 0.0015
8, 5, 40, 0.002
8, 5, 40, 0.0025
8, 5, 40, 0.003
8, 5, 40, 0.0035
8, 5, 40, 0.004
8, 5, 40, 0.005
8, 5, 40, 0.006
8, 5, 40, 0.007
8, 5, 40, 0.008
8, 5, 40, 0.009
8, 5, 40, 0.01
8, 5, 40, 0.02
8, 5, 40, 0.03
8, 5, 40, 0.04
8, 5, 40, 0.05
8, 5, 40, 0.06
8, 5, 40, 0.08
8, 5, 40, 0.1
\end{verbatim}

The experiments for the algorithm are executed with the command:

\begin{Schunk}
\begin{Sinput}
> experiment("../extdata/8-experiment")
\end{Sinput}
\end{Schunk}

\subsection{SWMTF}

This method uses a single parameter for the parameter $\alpha$ defining the significance
level of the tests. The file named \textbf{11-experiment} contains the configurations
tests looking for the best value of this parameter:

\begin{verbatim}
1000,4
100
15, 0.001, 0.0015, 0.002, 0.0025, 0.003, 0.0035, 0.004, 0.0045, 0.005, 0.0055,
0.006, 0.0065, 0.007, 0.0075, 0.008, 0.0085, 0.009, 0.0095, 0.01, 0.02, 0.03,
0.04, 0.05, 0.06, 0.07, 0.08, 0.09, 0.1
\end{verbatim}

The experiments for the algorithm are executed with the command:

\begin{Schunk}
\begin{Sinput}
> experiment("../extdata/15-experiment")
\end{Sinput}
\end{Schunk}

The results show that the best value for $\alpha$ is $0.01$.

\subsection{SWMTFIn}

This algorithm requires the following parameters:

\begin{itemize}
\item $n$: size of the active window
\item $\alpha$: value to consider for statistical tests
\end{itemize}

The content of the file \textbf{21-experiment} shows the configurations
tested during the experimental work:

\begin{verbatim}
1000,4
100
21, 5, 40, 0.001
21, 5, 40, 0.0015
21, 5, 40, 0.002
21, 5, 40, 0.0025
21, 5, 40, 0.003
21, 5, 40, 0.0035
21, 5, 40, 0.004
21, 5, 40, 0.005
21, 5, 40, 0.006
21, 5, 40, 0.008
21, 5, 40, 0.01
21, 5, 40, 0.02
21, 5, 40, 0.03
21, 5, 40, 0.04
21, 5, 40, 0.05
21, 5, 40, 0.06
21, 5, 40, 0.08
21, 5, 40, 0.1
\end{verbatim}

\subsection{SWMTB}

This algorithm receives as input a single parameter used for the statistical
tests: $\alpha$. The experiments executed in order to find the best value
for it are defined in \textbf{23-experiment.R}:

\begin{verbatim}
1000,4
100
23, 0.001, 0.0015, 0.002, 0.0025, 0.003, 0.0035, 0.004, 0.0045, 0.005, 0.0055,
0.006, 0.0065, 0.007, 0.0075, 0.008, 0.0085, 0.009, 0.0095, 0.01, 0.02, 0.03,
0.04, 0.05, 0.06, 0.07, 0.08, 0.09, 0.1
\end{verbatim}

The optimal value is $\alpha=0.06$.

\subsection{SWMTBIn}

This method uses two parameters:

\begin{itemize}
\item $n$: size of active window
\item $\alpha$: value to use for statistical tests
\end{itemize}

The file named \textbf{24-experiment} contains all the tested parameterizations:

\begin{verbatim}
1000,4
100
24, 5, 40, 0.001
24, 5, 40, 0.0015
24, 5, 40, 0.002
24, 5, 40, 0.0025
24, 5, 40, 0.003
24, 5, 40, 0.0035
24, 5, 40, 0.004
24, 5, 40, 0.005
24, 5, 40, 0.006
24, 5, 40, 0.008
24, 5, 40, 0.01
24, 5, 40, 0.02
24, 5, 40, 0.03
24, 5, 40, 0.04
24, 5, 40, 0.05
24, 5, 40, 0.06
24, 5, 40, 0.08
24, 5, 40, 0.1
\end{verbatim}

\end{document}
